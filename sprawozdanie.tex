\documentclass[12pt,a4paper,twoside]{article}

\usepackage{amsmath,amssymb}
\usepackage[utf8]{inputenc}                                      
\usepackage[OT4]{fontenc}      
%\usepackage[T1]{fontenc}                            
\usepackage[polish]{babel}                           
\selectlanguage{polish}
\usepackage{indentfirst} 
\usepackage[dvips]{graphicx}
\usepackage{tabularx}
\usepackage{color}
\usepackage{hyperref} 
\usepackage{fancyhdr}
\usepackage{listings}
\usepackage{booktabs}
\usepackage{ifpdf}
\usepackage{mathtext} % polskie znaki w trybie matematycznym
%\makeindex  % utworzenie skorowidza (w dokumencie pdf)
\usepackage{lmodern}
%\usepackage[osf]{libertine}
\usepackage{filecontents}
\usepackage{ifthen}
\usepackage{spverbatim}


\usepackage{tikz}
\usetikzlibrary{arrows}


\newcounter{nextYear}
\setcounter{nextYear}{\the\year}
\stepcounter{nextYear}

% rozszerzenie nieco strony
%\setlength{\topmargin}{-1cm} \setlength{\textheight}{24.5cm}
%\setlength{\textwidth}{17cm} \addtolength{\hoffset}{-1.5cm}
%\setlength{\parindent}{0.5cm} \setlength{\footskip}{2cm}
%\linespread{1.2} % odstep pomiedzy wierszami


%%%% ZYWA PAGINA %%%%%%%%%%%
\newcommand{\tl}[1]{\textbf{#1}} 
\pagestyle{fancy}
\renewcommand{\sectionmark}[1]{\markright{\thesection\ #1}}
\fancyhf{} % usuwanie bieżących ustawień
\fancyhead[LE,RO]{\small\bfseries\thepage}
\fancyhead[LO]{\small\bfseries\rightmark}
\fancyhead[RE]{\small\bfseries\leftmark}
\renewcommand{\headrulewidth}{0.5pt}
\renewcommand{\footrulewidth}{0pt}
\addtolength{\headheight}{0.5pt} % pionowy odstęp na kreskę
\fancypagestyle{plain}{%
\fancyhead{} % usuń p. górne na stronach pozbawionych numeracji
\renewcommand{\headrulewidth}{0pt} % pozioma kreska
}

%%%%%   LISTINGI %%%%%%%%
% ustawienia listingu programow

\lstset{%
language=C++,%
commentstyle=\textit,%
identifierstyle=\textsf,%
keywordstyle=\sffamily\bfseries, %
%captionpos=b,%
tabsize=3,%
frame=lines,%
numbers=left,%
numberstyle=\tiny,%
numbersep=5pt,%
breaklines=true,%
morekeywords={pWezel,Wezel,string,ref,params_result},%
escapeinside={(*@}{@*)},%
%basicstyle=\footnotesize,%
%keywords={double,int,for,if,return,vector,matrix,void,public,class,string,%
%float,sizeof,char,FILE,while,do,const}
}
%%%%%%%%%%%%%%%%%%%%%%%%%%%%%%%%%%%%%%%%%%%%%%%%%%%%%%%%%%%%%%%%%%%%%%%

%%%%%%%%%  NOTKI NA MARGINESIE %%%%%%%%%%%%%
% mala zmiana sposobu wyswietlania notek bocznych
\let\oldmarginpar\marginpar
\renewcommand\marginpar[1]{%
  {\linespread{0.85}\normalfont\scriptsize%
\oldmarginpar[\hspace{1cm}\begin{minipage}{3cm}\raggedleft\scriptsize\color{black}\textsf{#1}\end{minipage}]%    left pages
{\hspace{0cm}\begin{minipage}{3cm}\raggedright\scriptsize\color{black}\textsf{#1}\end{minipage}}% right pages
}%
}
% % % % % % % % % % % % % % % % % % % % % % % % % % % % % % % %

%%%% WYSWIETLANIE AKTUALNEGO ROKU AKADEMICKIEGO %%%%%%%%%%%
\newcounter{rok}
\newcommand{\rokakademicki}{%
   \setcounter{rok}{\number\year}%
   \ifthenelse{\number\month<10}%
   {\addtocounter{rok}{-1}}% rok akademicki zaczal sie w pazdzierniku poprzedniego roku
   {}%                       rok akademicki zaczyna sie w pazdzierniku tego roku
   \arabic{rok}/\addtocounter{rok}{1}\arabic{rok}
}
%%%%%%%%%%%%%%%%%%%%%%%%%%%%%%%%%%%%%%%


%%%% LISTA UWAG %%%%%%%%%
\usepackage{color}
\definecolor{brickred}      {cmyk}{0   , 0.89, 0.94, 0.28}

\makeatletter \newcommand \kslistofremarks{\section*{Uwagi} \@starttoc{rks}}
\newcommand\l@uwagas[2]
{\par\noindent \textbf{#2:} %\parbox{10cm}
   {#1}\par} \makeatother


\newcommand{\ksremark}[1]{%
   {{\color{brickred}{[#1]}}}%
   \addcontentsline{rks}{uwagas}{\protect{#1}}%
}

\newcommand{\comma}{\ksremark{przecinek}}
\newcommand{\nocomma}{\ksremark{bez przecinka}}
\newcommand{\styl}{\ksremark{styl}}
\newcommand{\ortografia}{\ksremark{ortografia}}
\newcommand{\fleksja}{\ksremark{fleksja}}
\newcommand{\pauza}{\ksremark{pauza `--', nie dywiz `-'}}
\newcommand{\kolokwializm}{\ksremark{kolokwializm}}
\newcommand{\cytowanie}{\ksremark{cytowanie}}

%%%%%%%%%%%%%%%%%%%%%%%%%
%%%%%%%%%%%%%%%%%%%%%%%%%
%%%%%%%%%%%%%%%%%%%%%%%%%
%%%%%%%%%%%%%%%%%%%%%%%%%
%%%%%%%%%%%%%%%%%%%%%%%%%
%%%%%%%%%%%%%%%%%%%%%%%%%
%%%%%%%%%%%%%%%%%%%%%%%%%
%%%%%%%%%%%%%%%%%%%%%%%%%
%%%%%%%%%%%%%%%%%%%%%%%%%
%%%%%%%%%%%%%%%%%%%%%%%%%
%%%%%%%%%%%%%%%%%%%%%%%%%
%%%%%%%%%%%%%%%%%%%%%%%%%



% autor:
\fancyhead[RE]{\small\bfseries Adam Fudala} % autor sprawozdania



%%%%%%%%%%% NO I ZACZYNA SIE SPRAWOZDANIE %%%%%%%%%%%

\begin{document}
\frenchspacing
\thispagestyle{empty}
\begin{center}
{\Large\sf Politechnika Śląska   % Alma Mater

Wydział Informatyki, Elektroniki i Informatyki

}

\vfill

 

\vfill\vfill

{\Huge\sffamily\bfseries Podstawy Programowania Komputerów\par}  

\vfill\vfill

{\LARGE\sf Kurier}   


\vfill \vfill\vfill\vfill

%%%%%%%%%%%%%%%%%%%%%%%%%%%%





\begin{tabular}{ll}
	\toprule
	autor                       & Adam Fudala    \\
	prowadzący                  &  dr. inż. Marcin Połomski \\
	rok akademicki              & \rokakademicki         \\
	kierunek                    & informatyka            \\
	rodzaj studiów              & SSI                    \\
	semestr                     & 1                      \\
	termin laboratorium         & poniedziałek, 08:30 -- 10:00 \\
	sekcja                      & 21            \\
	termin oddania sprawozdania & 2019-01-25            \\
	\bottomrule
	                            &
\end{tabular}

\end{center}

%%%%%%%%%%%%%%%%%%%%%%%%%%%%%%%%%%%%%%%%%%%%%%%%%%%%%%%%%%%%%%%%%%%%%%%%%
\cleardoublepage
%%%%%%%%%%%%%%%%%%%%%%%%%%%%%%%%%%%%%%%%%%%%%%%%%%%%%%%%%%%%%%%%%%%%%%%%%

%%%%%%%%%%%%%%%%%%%%%%%%%%%%%%%%%%%%%%%%%%%%%%%%%%%%%%%%%%%%%%%%%%%%%%%%%
\section{Treść zadania}
\marginpar{}
Kurier ma za zadanie zawieźć towar do klientów w różnych lokalizacjach i powrócić do miejsca, z którego
wyjechał. Kurier musi odwiedzić każdego klienta raz i tylko raz. Należy znaleźć zamkniętą najkrótszą drogę,
która umożliwia odwiedzenie wszystkich klientów. W pliku wejściowym zapisane są długości dróg pomiędzy
miastami. Drogi zapisane są w następujący sposób \begin{verbatim}(<klient A> - <klient B>: <odległość>)\end{verbatim}. Niektóre drogi
nie są symetryczne, tzn. jest pewna różnica między drogą tam a z powrotem. Zapis\begin{verbatim} (<klient C> -> <klient D> : <odległość CD>)\end{verbatim}, oznacza długość drogi jednokierunkowej od klienta C do klienta D. Poszczególne
drogi są rozdzielone przecinkami. Nie jest podana liczba dróg. Jeżeli nie jest możliwe wyznaczenie drogi,
program zgłasza odpowiedni komunikat. \newline 
Przykładowy plik wejściowy: \newline\newline
\begin{verbatim}
(1 - 2 : 4.5), (4 -> 3:
4.5),
(4 - 2: 0.4)\newline
\end{verbatim}
W pliku wynikowym należy zapisać drogę kuriera (kolejność odwiedzania klientów i długość drogi).
\indent Program uruchamiany jest z linii poleceń z wykorzystaniem następujących przełączników: \newline
\begin{tabular}{ll}
\indent \texttt{-i} & plik wejściowy  z drogami między klientami\\
\indent \texttt{-d} & plik wyjściowy  ze znalezioną drogą kuriera\\
\end{tabular}

%%%%%%%%%%%%%%%%%%%%%%%%%%%%%%%%%%%%%%%%%%%%%%%%%%%%%%%%%%%%%%%%%%%%%%%%%
\section{Analiza zadania}
\marginpar{}

Zagadnienie przedstawia problem wyznaczenia różnych dróg kuriera, gdzie przechodzi on między wszystkimi klientami tylko raz i wraca na początek trasy

\subsection{Struktury danych}
\marginpar{}
W programie wykorzystano listę podwieszaną. Lista nadrzędna przechowuje informację z \texttt{miastami początkowymi}. Lista nadrzędna zawiera wskaźnik na listę podrzędną, która przechowuje informacje o \texttt{mieście docelowym}  i \texttt{odległości}. Lista miast odwiedzonych przechowuje informację o \texttt{długości trasy} i wskazuje  na elementy zawierające odwiedzane po koleii \texttt{miasta}. Zdecydowałem, żeby użyć listy, ponieważ najlepiej odzwierciedlała rodzaj danych używanych w programie i pozwalała na wypisywanie klientów w określonej liście.





\subsection{Algorytmy}
\marginpar{}
Algorytm wykorzystuje rekurencję do wyznaczenia wszystkich tras z każdego wierzchołka grafu. Za każdym przejściem sprawdza, czy wszystkie wierzchołki już odwiedzono. Szuka miast w liście podwieszanej danego elementu i sprawdza czy te miasta znajdują się na liście odwiedzonych miast. Jeżeli miasta tam nie ma algorytm szuka go na liście miast początkowych i dodaje do listy odwiedzonych. W przeciwnym wypadku wyznacza nową trasę. Długość trasy jest porównywana z minimalną i nadpisywana.


%%%%%%%%%%%%%%%%%%%%%%%%%%%%%%%%%%%%%%%%%%%%%%%%%%%%%%%%%%%%%%%%%%%%%%%%%
\section{Specyfikacja zewnętrzna}
\marginpar{}

Program jest uruchamiany z linii poleceń. 
Przy wywoływaniu programu możliwe jest użycie przełączników   \texttt{-i} oraz \texttt{-d}\\ 
Po wykorzystaniu przełącznika  \texttt{-i} należy przekazać do programu nazwę pliku wejściowego. Po wykorzystaniu przełącznika  \texttt{-i} należy przekazać do programu nazwę pliku wejściowego. Domyślny format pliku to txt.
\begin{verbatim}
kurier.exe -i input.txt -d output.txt
\end{verbatim}

Program zapisuje najkrótszą drogę w pliku tekstowym w folderze zewnętrznym \texttt{pliki}. 
Pliki muszą mieć rozszerzenie .txt.


Uruchomienie programu z parametrem \texttt{-i} powoduje otwarcie pliku \texttt{plik.txt}  zawierającego odległości między poszczególnymi klientami. 


Uruchomienie programu z parametrem \texttt{-d} powoduje otwarcie pliku \texttt{plik.txt}  i zapisanie do niego ostatecznej trasy. 
Podanie nieprawidłowej nazwy pliku wejściowego powoduje wyświetlenie odpowiedniego komunikatu:
\begin{verbatim}
Nie znaleziono pliku wejściowego
\end{verbatim}
Podanie nieprawidłowej nazwy pliku wyjściowego powoduje wyświetlenie odpowiedniego komunikatu:
\begin{verbatim}
Nie znaleziono pliku wyjściowego
\end{verbatim}

Podanie danych niewystarczających do wyznaczenia trasy powoduje wyświetlenie komuniktu
\begin{verbatim}
Zbyt mało danych, aby wyznaczyć trasę
\end{verbatim}




\subsection{Ogólna struktura programu}
\marginpar{}
Wywoływana jest funkcja \lstinline|wczytajGraf|, która wczytuje dane z pliku do listy podwieszanej. 
Następnie w pętli wywoływana jest funkcja \lstinline|wyznaczNajlepszaTrase|. 
Funkcja przechodzi rekurencyjnie przez listę, wpisuje do listy miast odwiedzonych każdą trasę. 
Zapisywana jest tylko najkrótsza trasa. Po jej wyznaczeniu funkcja \lstinline|zapiszDoPliku| zapisuje dane do pliku.



\subsection{Szczegółowy opis typów i funkcji}

Szczegółowy opis typów i funkcji zawarty jest w załączniku. 

\section{Testowanie}
\marginpar{}

Kiedy podałem pusty plik, program wyświetlił komunikat \texttt{zbyt mało danych, aby wyznaczyć trasę}. Kiedy podałem takie drogi, że nie dało się wyznaczyć trasy komunikat był taki sam. 
Gdy zamiast pliku wejściowego podałem losowy ciąg znaków, program wyświetlił komunikat \texttt{Nie znaleziono pliku wejściowego}, gdy podobnie zrobiłem z plikiem wyjściowym komunikat mówił \texttt{Nie znaleziono pliku wyjściowego}.

Program został sprawdzony pod kątem wycieków pamięci przy użyciu biblioteki cstdlib i wbudowanego narzędzia visual studio.

%\section{Uzyskane wyniki}
%Jeżeli zadanie tego wymaga, wyniki można przestawić w~różny sposób, np. w tabeli ({\it vide} tabela \ref{tab:1}), czy na~rysunku ({\it vide} rycina \ref{fig:rysunek}) albo po prostu opisać uzyskane wyniki.
%



\section{Wnioski}
\marginpar{}
Największe trudności sprawiło mi stworzenie poprawnie działającego algorytmu, który przeszedłby po wszystkich możliwych trasach oraz ustawienie parametrów wejściowych łącznie z zabezpieczeniami. Przy przepisywaniu argumentów wejściowych, zorientowałem się, że straciłem wskaźnik na char* z plikiem wejściowym i wyjściowym, więc wpisałem tam nazwę pliku litera po literze. Projekt nauczył mnie głównie pracy na listach i gospodarki pamięcią. W przygotowywaniu projektu bardzo pomogła mi wiedza, którą nabyłem na laboratorium i wykładach.  

 
\begin{filecontents}{bibliografia.bib}

}
\end{filecontents}




 
\cleardoublepage

\rule{0cm}{0cm}

\vfill

\begin{center}
\Huge\bfseries Dodatek\\Szczegółowy opis typów i~funkcji\par
\end{center}

\vfill 

\rule{0cm}{0cm}

\end{document}
% Koniec wieńczy dzieło.